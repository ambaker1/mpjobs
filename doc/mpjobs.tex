\documentclass{article}

% Input packages & formatting
% Packages

% Math packages
\usepackage{amsmath} % Extended math functions
\usepackage{amssymb} % Extended math symbols (loads in amsfonts)
\usepackage{bm} % Bold math symbols
\usepackage{mathtools}

% Figure packages
\usepackage{caption} % Caption formatting for university standard
\usepackage{graphicx} % includegraphics command
\usepackage{subcaption} % Subfigures
\usepackage[section]{placeins} % Place floats in section
\usepackage{wrapfig}

% Table packages
\usepackage{booktabs} % Better tables
\usepackage{bigstrut} % Merged table cells
\usepackage{longtable} % Tables which overflow into next page
\usepackage{array}
\usepackage{colortbl} % Color table cells
\usepackage{makecell}
\usepackage{multirow}

% Fonts
\usepackage{lmodern} % Use latin modern rather than computer modern. Better for font encoding.
\usepackage[T1]{fontenc} % Allow text to be searchable in output

% Other packages
\usepackage{appendix} % Appendix environment
\usepackage{nextpage} % Cleartooddpage command
%\usepackage[square,comma,sort,numbers]{natbib} % Reference formatting
\usepackage{setspace} % Line spacing
\usepackage{listings} % Display code with syntax highlighting
\usepackage{upquote} % Vertical quotes in verbatim
\usepackage{xcolor} % Colors
\usepackage{titlesec} % Header spacing
\usepackage{xparse} % for tcolorbox
\usepackage[listings]{tcolorbox} % Colored boxes for highlighting syntax
\tcbuselibrary{breakable}
\tcbuselibrary{skins}
\usepackage{enumitem} % better enumerate/itemize options
\usepackage{fancyhdr}
\usepackage{multicol}
\usepackage{ifthen}
\usepackage{xstring}

% Table of contents
\usepackage{imakeidx} % Index page
\usepackage{tocloft} % Control of table of contents
\usepackage[nottoc]{tocbibind} % Adds bibliography, table of tables, table of figures, to table of contents
\usepackage[bookmarks,linktocpage,hidelinks]{hyperref} % Hyperlinks for sections, figures, etc.

% Formatting
% Page format
\setlength{\oddsidemargin}{0.00in}  % Left side margin for odd numbered pages
\setlength{\evensidemargin}{0.00in} % Right side margin for even numbered pages
\setlength{\topmargin}{0.00in}      % Top margin
\setlength{\headheight}{0.20in}     % Header height
\setlength{\headsep}{0.20in}        % Separation between header and main text
\setlength{\topskip}{0.00in}        % Top skip
\setlength{\textwidth}{6.50in}      % Width of the text
\setlength{\textheight}{8.50in}     % Height of the text
\setlength{\footskip}{0.50in}       % Foot skip
\setlength{\parindent}{0.00in}      % First line indentation
\setlength{\parskip}{6pt}        % Space between two paragraphs

% Captions (figures, tables, etc.)
\setlength{\floatsep}{\parskip}          % Space left between floats.
\setlength{\textfloatsep}{\floatsep}   % Space between last top float
% or first bottom float and the text
\setlength{\intextsep}{\floatsep}      % Space left on top and bottom
% of an in-text float
\setlength{\abovecaptionskip}{0.1in plus 0.25in}  % Space above caption
\setlength{\belowcaptionskip}{0.1in plus 0.25in}  % Space below caption
\setlength{\captionmargin}{0.50in}     % Left/Right margin for caption
\setlength{\abovedisplayskip}{0.00in plus 0.25in} % Space before Math stuff
\setlength{\belowdisplayskip}{0.00in plus 0.25in} % Space after Math stuff
\setlength{\arraycolsep}{0.10in}       % Gap between columns of an array
\setlength{\jot}{0.10in}                % Gap between multiline equations
\setlength{\itemsep}{0.10in}           % Space between successive items

% Counters (no section numbering)
\setcounter{tocdepth}{3}
\setcounter{secnumdepth}{0}

% Spacing
\setstretch{1.5}

\titlespacing*{\section}{0cm}{6pt}{6pt}[0cm]
\titlespacing*{\subsection}{0cm}{6pt}{6pt}[0cm]
\titlespacing*{\subsubsection}{0cm}{6pt}{6pt}[0cm]

\titleformat{\section}
{\sffamily\huge}{}{0pt}{\titlerule\vspace{-0.2cm}}
\titleformat{\subsection}
{\sffamily\itshape\Large}{}{0pt}{}

% Macro for syntax
\newtcolorbox{syntax}{
    size=small,
    sharp corners,
    colframe=black,
    colback=yellow,
    fontupper=\bfseries\ttfamily
}

% Macro for argument table
\newenvironment{args}{
    \begin{tabular}{>{\bfseries\ttfamily}p{0.25\linewidth} p{0.69\linewidth}}
    }{
    \end{tabular}\par
    \vspace{0.5\baselineskip}
}

% Note: Requires packages "listing", "xcolor", and "textcomp"
\lstdefinelanguage{verbatim}{
    basicstyle=\ttfamily\small,
    xleftmargin=9pt,
    xrightmargin=9pt,
    columns=fullflexible,
    keepspaces=true,
    breaklines=true
}

% Example code
\AtBeginDocument{
\newtcolorbox[blend into=listings]{example}[2][]{
    colback=blue!3!white,
    colframe=black,
    colbacktitle=blue!15!white,
    coltitle=black,
    sharp corners,
    enhanced,
    breakable,
    size=small,
    before upper={
        \setstretch{1.0}\lstset{language=verbatim}\vspace{3pt}\textsf{\textit{Code:}}
    },
    subtitle style={
        colback=blue!20!white,
        fonttitle=\sffamily
    },
    before lower={
        \setstretch{1.0}\lstset{language=verbatim}\vspace{3pt}\textsf{\textit{Output:}}
    },
    fonttitle=\sffamily,
    title={#2},
    #1
}
}

% Links to sub and subsub commands - optional boolean argument, default true. if false, only displays subcmd.

% Commands (and command ensembles)
\newcommand{\command}[1]{\protect\hypertarget{#1}{#1}\index{#1}}
\newcommand{\subcommand}[2]{\protect\hypertarget{#1 #2}{#1 #2}\index{#1!#2}}
\newcommand{\cmdlink}[1]{\protect\hyperlink{#1}{\textit{#1}}}
\newcommand{\subcmdlink}[3][1]{\protect\hyperlink{#2 #3}{\ifnum#1=1\relax\textit{#2 #3}\else\textit{#3}\fi}}

% Methods (first arg is class)
\newcommand{\method}[2]{\protect\hypertarget{$#1Obj #2}{\$#1Obj #2}\index{#1 methods!#2}}
\newcommand{\methodlink}[3][1]{\protect\hyperlink{$#2Obj #3}{\ifnum#1=1\relax\textit{\$#2Obj #3}\else\textit{#3}\fi}}

% Macros for figure/table names
\newcommand{\fig}{\figurename\ }
\newcommand{\figs}{\figurename s }
\newcommand{\tbl}{\tablename\ }
\newcommand{\tbls}{\tablename s }
\newcommand{\eq}{Eq. }
\newcommand{\eqs}{Eqs. }
\renewcommand{\lstlistingname}{Example}% Listing -> Example
\renewcommand{\lstlistlistingname}{List of \lstlistingname s}% List of Listings -> List of Examples
\newcommand{\ex}{Example }
\newcommand{\exs}{Examples }
\newcommand{\var}[1]{\texttt{\textbf{\$#1}}}

% Header/footer
\renewcommand{\headrulewidth}{0pt}

% Changes to hyperlinks (URLs)
\renewcommand\UrlFont{\color{blue}\rmfamily}

% New column type 
% https://tex.stackexchange.com/questions/75717/how-can-i-mix-itemize-and-tabular-environments
\newcolumntype{L}{>{\labelitemi~~}l<{}}
\newcommand{\version}{0.1.1}

\renewcommand{\cleartooddpage}[1][]{\ignorespaces} % single side
\newcommand{\caret}{$^\wedge$}

% Other macros
\renewcommand{\^}[1]{\textsuperscript{#1}}
\renewcommand{\_}[1]{\textsubscript{#1}}

\title{\Huge OpenSeesMP Job Board Framework (mpjobs)\\\small Version \version}
\author{Alex Baker\\\small\url{https://github.com/ambaker1/mpjobs}}
\date{\small\today}
\begin{document}
\maketitle
\begin{abstract}
OpenSeesMP provides basic MPI message-passing commands such as \textit{getPID}, \textit{getNP}, \textit{send}, \textit{recv}, and \textit{barrier}, which can be used to run parametric studies in parallel. 
The mpjobs package provides an abstract framework for running parametric studies in series and parallel, eliminating the need to deal directly with the message passing commands.
Additionally, it formalizes and organizes the parametric study in a way that supports save-state functionality, allowing the analyst to revisit a study and analyze it further.
\end{abstract}
\clearpage
\section{General}
The main command for the job module is \cmdlink{jobBoard}.
All other job board commands must be called within the body of \cmdlink{jobBoard}.
This command encapsulates an ensemble of subcommands used for creating, manipulating, running, and accessing jobs.
This command can be called in series or in parallel (OpenSeesMP).
If it is called in parallel, it must be called by all processes, but the body will only be evaluated on process 0.
If the main process is aborted, any progress on active jobs will be lost, and will be reset the next time the job board is opened.
To stop the job board without aborting active jobs, enter ``stop'' on the command line. 
This will signal to the main process to stop assigning/running new jobs, and will proceed to finish all active jobs and exit.
The ``-timeout'' option does the same thing as entering ``stop'', but does so after a specified duration.
\begin{syntax}
\command{jobBoard} <-debug> <-wipe> <-timeout \$duration> \$path \$body
\end{syntax}
\begin{args}
-wipe & Option to wipe the job board (\cmdlink{wipeJobs}) before loading job data. \\
-debug & Option to display debug information. \\
-timeout \$duration & Option to time-out the job board analysis. \\ & Duration must be in format ``HH:MM:SS''. \\
\$path & Folder containing job board files. \\
\$body & Body to evaluate, with access to job board.
\end{args}

When \cmdlink{jobBoard} is called, it creates the job board folder \texttt{\$path}, where job data is stored in files. 
The specific files created by the job board framework are listed below, where \texttt{\$jobTag} is a non-negative integer: 
\begin{table}[!htb]
\centering
\caption{Job board file descriptions}
\label{tbl:job_board_files}
\begin{tabular}{ll}
\toprule
Filename & Description \\
\midrule
INPUTS & File containing Tcl list of job inputs. \\
SESSION & File containing system PID of process accessing job board. \\
STATUS & File containing status codes of all jobs. \\
\$jobTag.tcl & Tcl script which runs job \$jobTag. \\
\$jobTag.log & Log of job \$jobTag stderr channel. \\
\$jobTag.dat & Results of job \$jobTag. \\
\bottomrule
\end{tabular}
\end{table}

\clearpage

\subsection{Creating Jobs}
The command \cmdlink{makeJob} creates a new job in the job board and returns the new job tag, unless if the given inputs exist, in which case it will simply return the matching job tag.
Job tags start at 0 and increment each time a new job is created.
The input file for a job must return a dictionary of results (or blank).
\begin{syntax}
\command{makeJob} <\$inputDir> \$inputFile \$var1 \$value1 ...
\end{syntax}
\begin{args}
\$inputDir & Input directory for job. Either an absolute directory or relative to job board. Default ``..'', or parent directory to job board. \\
\$inputFile & Input file for job \\
\$var1 \$value1 ... & Variable names and values to pass to the input file. Variable names ``folder'', ``filename'', and ``status'' are reserved. 
\end{args}

\subsection{Running Jobs}
The command \cmdlink{runJobs} runs all jobs with status code 0 (available) in the job board in a separate instance of OpenSees, using the OpenSees executable on the same path as the current executable.
If jobs are specified that have non-zero status codes, it will skip them.
In series mode, it will run the jobs sequentially, and return as soon as all jobs are complete. 
In parallel mode, it will assign the jobs to the workers using a load-sharing scheme, and will return as soon as all jobs are assigned. 
In both series and parallel, if ``stop'' is entered to the main process stdin channel or if the timeout duration has been reached, it will not run or assign any jobs, but will simply return.
\begin{syntax}
\command{runJobs} <\$jobTags>
\end{syntax}
\begin{args}
\$jobTags & Jobs to run. Default ``-all'', for all available.
\end{args}
\begin{example}{Basic Use}
\begin{lstlisting}
jobBoard MyStudy {
    foreach a $aList {
        foreach b $bList {
            makeJob InputFile.tcl a $a b $b
        }
    }
    runJobs
}
\end{lstlisting}
\end{example}

\clearpage
\section{Job Utilities}

\subsection{Removing all Jobs}
The command \cmdlink{wipeJobs} removes all jobs and job data from the job board.
If any jobs are active, it will wait for them to finish.
This is also called by \cmdlink{jobBoard} if option -wipe is used.
\begin{syntax}
\command{wipeJobs}
\end{syntax}

\subsection{Resetting Jobs}
The command \cmdlink{resetJobs} resets jobs to status 0, or ``available''. 
If jobs are specified that are already available, it will skip them.
If active jobs are specified, it will wait for them to finish first.
The intended use of this command is to re-run analyses when the analysis file changes.
\begin{syntax}
\command{resetJobs} <\$jobTags>
\end{syntax}
\begin{args}
\$jobTags & Jobs to reset. Default ``-all'', for all jobs with non-zero status.
\end{args}
\begin{example}{Re-running a parametric analysis}
\begin{lstlisting}
jobBoard MyStudy {
    resetJobs
    runJobs
}
\end{lstlisting}
\end{example}

\clearpage

\subsection{Waiting for Jobs}
The command \cmdlink{waitForJobs} waits for active jobs to be complete, and then updates the job statuses.
This is essentially a ``barrier'' call for worker processes.
\begin{syntax}
\command{waitForJobs} <\$jobTags>
\end{syntax}
\begin{args}
\$jobTags & Jobs to wait for. Default ``-all'', or all active jobs.
\end{args}

\subsection{Updating Jobs}
The command \cmdlink{updateJobs} updates the status of jobs running in parallel.
This is essentially the asynchronous version of \cmdlink{waitForJobs}, and is especially useful for dynamic parametric parallel studies.
\begin{syntax}
\command{updateJobs} <\$jobTags>
\end{syntax}
\begin{args}
\$jobTags & Jobs to update. Default ``-all'', or all active jobs.
\end{args}

\begin{example}{Asynchronous updating of job information}
\begin{lstlisting}
jobBoard MyStudy {
    foreach x {1 2 3} {
        makeJob MyFile x $x
    }
    runJobs
    # Asynchronously update job statuses and process completed jobs.
    while {[llength [getJobTags -active]] == 0} {
        updateJobs
        foreach jobTag [getJobTags -complete] {
            puts "Job $jobTag complete"
        }
        after 100; # Add a little buffer
    }
}
\end{lstlisting}
\end{example}

\clearpage
\section{Job Information Queries}
In addition to being able to create jobs and run them, the job board framework provides convenient routines to query current job board information. 

\subsection{Get List of Jobs}
The command \cmdlink{getJobTags} returns either all the jobs in a job board or a subset specified by status or pattern, sorted in increasing order.
If conflicting options are specified, the last options take precedence.
\begin{syntax}
\command{getJobTags} <-all> <-available> <-active> <-complete> <-failed> <\$codes>
\end{syntax}
\begin{args}
-all & All job tags (default). \\
-available & Only available jobs (status code 0). \\
-active & Only active jobs (status code 1). \\
-complete & Only completed jobs (status code 2). \\
-failed & Only failed jobs (status code 3). \\
\$codes & List of status codes. 
\end{args}

\subsection{Total Number of Jobs}
The total number of created jobs can be queried with the command \cmdlink{getJobCount}. 
\begin{syntax}
\command{getJobCount}
\end{syntax}

\subsection{Get Job Inputs}
Job inputs can be queried with \cmdlink{getJobInputs}. 
Returns a dictionary, with keys ``folder'', ``filename'', and any defined input variables.
\begin{syntax}
\command{getJobInputs} \$jobTag
\end{syntax}
\begin{args}
\$jobTag & Integer job tag.
\end{args}
\clearpage
\subsection{Get Job Status Code}
The command \cmdlink{getJobStatus} returns the status of a job. 
Note that \cmdlink{waitForJobs} or \cmdlink{updateJobs} must be called to update active job statuses.
\begin{syntax}
\command{getJobStatus} \$jobTag
\end{syntax}
\begin{args}
\$jobTag & Integer job tag.
\end{args}
Below are the job status codes and their meanings: 

\begin{tabular}{l l}
\quad 0 & Job is available to run. \\
\quad 1 & Job is active. \\
\quad 2 & Job completed successfully. \\
\quad 3 & Job completed with error.
\end{tabular}

\subsection{Get Job Results}
Job results can be queried with \cmdlink{getJobResults}. 
If the job status is < 2, it will return blank. 
If the job status is 2, it will return a dictionary of results.
If the job status is 3, it will return the option dictionary detailing the cause of the error.
\begin{syntax}
\command{getJobResults} \$jobTag
\end{syntax}
\begin{args}
\$jobTag & Name of job to query results.
\end{args}

\subsection{Get Job Table}	
The command \cmdlink{getJobTable} returns a Tda table with rows corresponding to jobs, and columns corresponding to job inputs, status, and results. 
Column headers are as follows: ``jobTag'', ``status'', ``folder'', ``filename'', any input variables, and any output variables. 

\begin{syntax}
\command{getJobTable} <\$jobTags>
\end{syntax}
\begin{args}
\$jobTags & Job tags to set as table keys. Default ``-all'' for all jobs.
\end{args}

Tda is available here: \url{https://github.com/ambaker1/Tda}.

\clearpage
\section{Example Application}
For this example, the displacement and moment of an elastic cantilever column was investigated for different geometry parameters. 
Note that the input variables ``L'' and ``I'' have defaults set by using the ``vutil'' package \textit{default} command. Also note the definition of a result dictionary at the end of the file.
\begin{example}[label=ex:job_cantilever]{Elastic cantilever column (Cantilever.tcl)}
\begin{lstlisting}
# Elastic Cantilever Column
package require tin
tin import default from vutil
model BasicBuilder -ndm 2 -ndf 3

# Variables (with defaults)
default L 10.0; # in
default I 1000.0; # in^4
set A 100.0; # in^2 (should not affect results)
set E 29000.0; # ksi

# Define nodes
node 1 0 0
node 2 $L 0
fix 1 1 1 1

# Define element
geomTransf Linear 1
element elasticBeamColumn 1 1 2 $A $E $I 1

# Add load
timeSeries Linear 1
pattern Plain 1 1 {
    load 2 0 1 0
}

# Setup analysis
constraints Plain
numberer RCM
system BandGeneral
test NormDispIncr 1.0e-8 6 
algorithm Newton
integrator LoadControl 1.0
analysis Static

# Perform analysis
analyze 10

# Return results
dict set results disp [expr {double([nodeDisp 2 2])}]
dict set results moment [expr {double([localForce 1 3])}]
return $results; # This is required for the job board framework
\end{lstlisting}
\end{example}
For this example, the length and moment of inertia of the column were varied, and the effect on the end displacement and cantilever moment were investigated, and the results are shown in \tbl\ref{tbl:job_cantilever_results}. Note the use of the ``tda'' package command \textit{writeTable}.

\begin{example}{Simple parameter study}
\begin{lstlisting}
# Parameter study of cantilever column
package require tin
tin import mpjobs
tin import writeTable from tda
jobBoard -wipe CantileverStudy {
    foreach L {10 20 30} {
        foreach I {100 200 300} {
            makeJob Cantilever.tcl L $L I $I
        }
    }
    runJobs
    waitForJobs
    writeTable CantileverStudy/results.csv [getJobTable]
}
\end{lstlisting}
\end{example}
\begin{table}[htbp]
\centering
\caption{Cantilever column parametric study results}
\begin{tabular}{rrllrrrr}
    \toprule
    jobTag & status & folder & filename       &  L &   I &     disp & moment \\
    \midrule
         0 &      2 & ..     & Cantilever.tcl & 10 & 100 & 0.001149 &   -100 \\
         1 &      2 & ..     & Cantilever.tcl & 10 & 200 & 0.000575 &   -100 \\
         2 &      2 & ..     & Cantilever.tcl & 10 & 300 & 0.000383 &   -100 \\
         5 &      2 & ..     & Cantilever.tcl & 20 & 100 & 0.009195 &   -200 \\
         6 &      2 & ..     & Cantilever.tcl & 20 & 200 & 0.004598 &   -200 \\
         7 &      2 & ..     & Cantilever.tcl & 20 & 300 & 0.003065 &   -200 \\
        10 &      2 & ..     & Cantilever.tcl & 30 & 100 & 0.031034 &   -300 \\
        11 &      2 & ..     & Cantilever.tcl & 30 & 200 & 0.015517 &   -300 \\
        12 &      2 & ..     & Cantilever.tcl & 30 & 300 & 0.010345 &   -300 \\
    \bottomrule
\end{tabular}%
\label{tbl:job_cantilever_results}
\end{table}
\end{document}
